%! Author = jonathan
%! Date = 5/25/25
\begin{abstract}
    Distributed Machine Learning (DML) is increasingly recognized as a communication-bound workload,
    with most existing work aiming to alleviate this bottleneck through communication--computation overlap.
    However, current approaches---largely reliant on CPU-managed operator scheduling and
    synchronous communication collectives---leave significant performance on the table.

    This thesis identifies, analyzes and addresses the inefficiencies arising from the
    interplay between CPU-driven collective communication and highly parallel GPU computation.
    We demonstrate that the standard bulk-synchronous communication model underutilizes GPU interconnect bandwidth
    and is especially vulnerable to straggler-induced performance degradation.
    Focusing on dynamic workloads such as Mixture-of-Experts (MoE), we highlight two critical bottlenecks.
    First, we observe \emph{payload inefficiency} at the application layer,
    where existing collective primitives force unnecessary padding of GPU network buffers.
    Second, we expose how CPU-driven execution limits the exploitation of \emph{task locality}
    and introduces artificial synchronization barriers across distributed GPU tasks.

    To overcome these limitations, we propose a model of \emph{complete GPU residency},
    where inter-GPU communication is integrated directly into GPU kernels.
    We realize this vision in \sysname: a persistent,
    in-kernel, actor-style operating system with packet switching
    that enables complete operator fusion for Distributed MoE (DMoE) into a \emph{single kernel},
    the first of its kind.
    \sysname features a modular, message-driven architecture that supports
    lockless execution across tens of thousands of GPU threads and across distributed GPUs as well.
    We demonstrate how \sysname addresses the all-to-all communication bottleneck in expert parallelism
    and enables high-throughput, GPU-initiated communication.
    Evaluated against state-of-the-art distributed MoE frameworks,
    \sysname achieves up to \textbf{9}$\times$ higher GPU utilization, \textbf{6}$\times$ lower latency,
    \textbf{5.7}$\times$ higher throughput, and \textbf{4}$\times$ better overlap efficiency
    compared to state-of-the-art baselines—despite using FP32 while baselines use FP16.
\end{abstract}