%! Author = jonathan
%! Date = 5/27/25
\chapter{Conclusion}\label{ch:conclusion}
This thesis introduces \sysname,
the first system to fuse the entire Mixture-of-Experts (MoE)
operator into a single, persistent GPU kernel.
We show that prevailing MoE implementations
suffer from two critical inefficiencies:
(1) CPU-managed synchronous communication that leads to underutilized
interconnects and (2) fragmented execution via multiple GPU kernels,
introducing overhead and synchronization delays.

In contrast, \sysname embraces a model of
GPU autonomy by embedding computation, communication, and scheduling within a unified kernel.
It leverages actor-style concurrency,
warp specialization, and asynchronous (R)DMA to
achieve fine-grained communication–computation overlap.

Our evaluation demonstrates up to \textbf{15$\times$ speedup}
over state-of-the-art systems, up to  \textbf{4$\times$} improved GPU utilization, and
greater payload efficiency, especially in dynamic workloads like MoE.
\sysname challenges the dominant execution paradigms in distributed deep learning
and presents a compelling template for building future GPU-native systems.

While several limitations remain---including programming complexity and lack of FP16 support---this
work lays the groundwork for a new era of \emph{in-kernel distributed computation}.
Future systems may build upon this foundation to enable kernel fusion for
entire training pipelines, ushering in a design shift from CPU orchestration to
\textbf{fully autonomous GPU execution}.